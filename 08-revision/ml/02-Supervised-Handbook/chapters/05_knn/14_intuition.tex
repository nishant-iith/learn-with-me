\chapter{KNN Intuition}

\section{Introduction}
\begin{quote}
    "You are the average of the five people you spend the most time with." — Jim Rohn
\end{quote}
This quote perfectly summarizes \textbf{K-Nearest Neighbors (KNN)}.
It is a simple, non-parametric algorithm used for both Classification and Regression. It assumes that similar things exist in close proximity.

\section{The Algorithm}
KNN is a \textbf{Lazy Learner}. It does not learn a discriminative function (like Logistic Regression or SVM) during training. Instead, it memorizes the training dataset.

\subsection{Steps}
For a new query point $x_q$:
\begin{enumerate}
    \item \textbf{Calculate Distances}: Measure the distance from $x_q$ to every point in the dataset.
    \item \textbf{Find Neighbors}: Sort distances and pick the top $K$ closest points.
    \item \textbf{Vote (Classification)}: Assign the most frequent class among the $K$ neighbors.
    \item \textbf{Average (Regression)}: Assign the mean value of the $K$ neighbors.
\end{enumerate}

\section{Distance Metrics}
How do we measure "closeness"?

\subsection{Euclidean Distance (L2 Norm)}
The straight-line distance. Default in most cases.
\[ d(p, q) = \sqrt{\sum (p_i - q_i)^2} \]

\subsection{Manhattan Distance (L1 Norm)}
The "Taxi-cab" distance (grid-like).
\[ d(p, q) = \sum |p_i - q_i| \]

\begin{remark}
    \textbf{Importance of Scaling}: Since KNN is based on distance, features with larger magnitudes (e.g., Salary vs Age) will dominate the calculation. \textbf{StandardScalization} (Mean=0, Std=1) is critical.
\end{remark}

\section{Choosing K (Hyperparameter)}
The value of $K$ controls the Bias-Variance tradeoff.

\begin{itemize}
    \item \textbf{Low K (e.g., 1)}: The model captures noise.
    \item \textbf{High K (e.g., N)}: The model predicts the majority class.
\end{itemize}

\textbf{Elbow Method}: Plot Error Rate vs $K$. The optimal $K$ is usually where the error drops and stabilizes (the "elbow").

\begin{figure}[htbp]
    \centering
    \includegraphics[width=0.6\textwidth]{knn_elbow_method_k_selection.png}
    \caption{Finding the optimal K using Cross-Validation.}
\end{figure}

\section{Interview Questions}

\begin{enumerate}
    \item \textbf{Why is KNN called a "Lazy Learner"?}
    \newline \textit{Answer:} Because it performs \textbf{zero} computation during the training phase. It only stores the data. All the heavy lifting (distance calculation) happens during prediction (inference).
    
    \item \textbf{What is the time complexity of KNN Prediction?}
    \newline \textit{Answer:} $O(N \cdot D)$, where $N$ is the number of training samples and $D$ is dimensions. This makes it very slow for large datasets.
    
    \item \textbf{How does K affect the decision boundary?}
    \newline \textit{Answer:}
    \begin{itemize}
        \item $K=1$: Very jagged, complex boundary (High Variance).
        \item $K=\text{Large}$: Smooth, simple boundary (High Bias).
    \end{itemize}
\end{enumerate}
